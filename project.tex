
Review about $CO_2$ forcing:
\bei
\item Integrating over all wavelengths in Fig.~\rf{olr} leads to a difference when the CO$_2$ abundance double from 280 to 560 ppm of 3.7W/m$^2$. That is, when the ghg concentration doubles, the OLR will decrease by 3.3 W/m$^2$. 
\Spng{olr}{Outgoing long wavelength radiation as a function of wavelength}
\item The deficit is logarithmically dependent on the concentration as indicated in Fig.~\rf{olrlog}:
\be
\Delta F = -3.7\log_2(c/c_0) W/m^2.\eql{log}\ee 
\Spng{olrlog}{Change in the emission at the top of the atmosphere in the 15 micron region as the CO$_2$ concentration goes up. The solid curve is the logarithmic fit as in \ec{log} .}
\item Taking the concentration today to be 420 ppm leads to a forcing of $3.7\log_2(420/280)=2 W/m^2$. This should have led to a temperature increase, using \ec{lambda} of 0.6$^\circ$C and an increase of 1.1$^\circ$C when the concentration doubles. In fact, we will see that the situation is much worse than that.
\eei

Feedbacks:
\bei
\item Initial forcing $F$ leads to initial temperature change
\be
\Delta T_0 = -F/\lambda_0.\ee
\item But then this temperature change leads to a new forcing $F_1=\lambda \Delta T_0$. For example, $\lambda$ could quantify the impact of the increase in water vapor due to the rise in temperature.  Note from Fig.~\rf{ghg} that this will decrease amount of radiation out and therefore lead to a positive forcing. The estimate of this forcing is $\lambda_{wv}=+1.6 W/m^2/K$.
\Spng{ghg}{Absorption as a function of wavelength for different green house gasses.}
\item This in turns leads to a new temperature change:
\be
\Delta T_1 = -F_1/\lambda_0 = -\frac{\lambda}{\lambda_0}\, \Delta T_0 = \frac{\lambda}{\lambda_0^2}\,F.\ee 
\item This leads to a new forcing $F_2=\lambda \Delta T_1 = (\lambda/\lambda_0)^2 F$ and a new temperature response
\be
\Delta T_2 = -F_2/\lambda_0 = -(\lambda^2/\lambda_0^3) F
\ee
\item so the total change in temperature is
\be
\Delta T_0 + \Delta T_1 + \Delta T_2 + \ldots = -\frac{F}{\lambda_0}\, \left[ 1 - \frac{\lambda}{\lambda_0} + \left( \frac{\lambda}{\lambda_0} \right)^2 + \ldots \right].
\ee
\item The sum is equal to $1/(1+x)$, so the total change in temperature is
\be
\Delta T = -\frac{F}{\lambda_0 + \lambda}.\ee
\item Using only water vapor, which turns out to be a good estimate because the other feedbacks cancel, leads to
\be
\Delta T = -\frac{F}{1.6 W/m^2} \, K\ee
\eei

\Spng{1990}{1990 IPCC prediction}
To date, the impact from $CO_2$ increase from 280 to 420 ppm then led to a forcing of 2W/m$^2$, which should have led to a temperature increase of about 1.2K. This is almost exactly what is seen in Figure~\rf{1990}, which was predicted back in 1990.

It also predicts that when the concentration doubles, the forcing will be 3.7 W/m$^2$, leading to a temperature increase of 3.7/1.6=2.3$^\circ$C. Fig.~\rf{ipcc6fig} shows some different scenarios for when this will happen.

\Sfig{ipcc6fig}{Different scenarios}

