%\documentclass[twocolumn,secnumarabic,amssymb, nobibnotes, aps, prd]{revtex4-2}
\documentclass[tsecnumarabic,amssymb, nobibnotes, aps, prd]{revtex4-2}

\usepackage{graphicx}
\usepackage{amsmath,mathtools,mathabx}
\usepackage{hyperref}


%%%%%%%%%%%%%%%
%%%%  Scott's macros
%%%%%%%%%%%%%%%

%%%%%%%%%%%%%%%
%%%% Equations
%%%%%%%%%%%%%%%
\def\be{\begin{equation}}
\def\ee{\end{equation}}
\def\bea{\begin{eqnarray}}
\def\eea{\end{eqnarray}}
\newcommand{\vs}{\nonumber\\}
\newcommand{\ec}[1]{Eq.~(\ref{eq:#1})}
\newcommand{\eec}[2]{Eqs.~(\ref{eq:#1}) and (\ref{eq:#2})}
\newcommand{\Ec}[1]{(\ref{eq:#1})}
\newcommand{\eql}[1]{\label{eq:#1}}


%%%%%%%%%%%%%%%
%%%%% Figures
%%%%%%%%%%%%%%%
\newcommand{\sfig}[2]{
\includegraphics[width=#2]{#1}
        }
\newcommand{\sfigr}[2]{
\includegraphics[angle=270,origin=c,width=#2]{#1}
        }
\newcommand{\Sfig}[2]{
   \begin{figure}[thbp]
   \begin{center}
    \sfig{Figures/#1.pdf}{0.7\columnwidth}
    \caption{{\small #2}}
    \label{fig:#1}
     \end{center}
   \end{figure}
}
\newcommand{\Sjpg}[2]{
   \begin{figure}[thbp]
   \begin{center}
    \sfig{Figures/#1.jpg}{0.6\columnwidth}
    \caption{{\small #2}}
    \label{fig:#1}
     \end{center}
   \end{figure}
}
\newcommand{\Sgif}[2]{
   \begin{figure}[thbp]
   \begin{center}
    \sfig{Figures/#1.gif}{0.6\columnwidth}
    \caption{{\small #2}}
    \label{fig:#1}
     \end{center}
   \end{figure}
}
\newcommand{\Spng}[2]{
   \begin{figure}[thbp]
   \begin{center}
    \sfig{Figures/#1.png}{0.8\columnwidth}
    \caption{{\small #2}}
    \label{fig:#1}
     \end{center}
   \end{figure}
}
\newcommand{\rf}[1]{\ref{fig:#1}}
\newcommand\bei{\begin{itemize}}
\newcommand\eei{\end{itemize}}
\newcommand\bee{\begin{enumerate}}
\newcommand\eee{\end{enumerate}}




\begin{document}
\title{Global Warming}
\author{Scott Dodelson}
\begin{abstract}
This paper traces a direct line from energy consumption to global warming.
\end{abstract}

\maketitle

\section{Introduction}

\section{Impact of $CO_2$ on global temperature}

\subsection{$CO_2$ forcing}
 {\it Radiative forcing} is the change in energy impinging on the surface of the Earth due to a certain effect, e.g., more sunlight; more albedo; more atmospheric absorption. The change is computed keeping all other components of the system (temperature, density, pressure, etc.) constant.  
\bei
\item Integrating over all wavelengths in Fig.~\rf{olr} leads to a difference when the CO$_2$ abundance double from 280 to 560 ppm of 3.7W/m$^2$. That is, when the ghg concentration doubles, the OLR will decrease by 3.3 W/m$^2$. 
\Spng{olr}{Outgoing long wavelength radiation as a function of wavelength}
\item The deficit is logarithmically dependent on the concentration as indicated in Fig.~\rf{olrlog}:
\be
\Delta F = -3.7\log_2(c/c_0) W/m^2.\eql{log}\ee 
\Spng{olrlog}{Change in the emission at the top of the atmosphere in the 15 micron region as the CO$_2$ concentration goes up. The solid curve is the logarithmic fit as in \ec{log} .}
\item Taking the concentration today to be 420 ppm leads to a forcing of $3.7\log_2(420/280)=2 W/m^2$. 
\item A given forcing $F$ leads to a temperature change:
\be
\Delta T = -F/\lambda_0
\ee
where
\be
\lambda_0=-3.2 W/m^2/K\eql{lambda}\ee
Note the sign: when the forcing is positive, more radiation impinges on the surface of the Earth so the temperature rises.
\item Therefore, the increase in $CO_2$ should have led to a temperature increase of 0.6$^\circ$C and an increase of 1.1$^\circ$C when the concentration doubles. In fact, we will see that the situation is much worse than that.
\eei

\subsection{Feedbacks}
\bei
\item Initial forcing $F$ leads to initial temperature change
\be
\Delta T_0 = -F/\lambda_0.\ee
\item But then this temperature change leads to a new forcing $F_1=\lambda \Delta T_0$. For example, $\lambda$ could quantify the impact of the increase in water vapor due to the rise in temperature.  Note from Fig.~\rf{ghg} that this will decrease amount of radiation out and therefore lead to a positive forcing. The estimate of this forcing is $\lambda_{wv}=+1.6 W/m^2/K$.
\Spng{ghg}{Absorption as a function of wavelength for different green house gasses. From \href{https://wattsupwiththat.com/2014/04/11/methane-the-irrelevant-greenhouse-gas/}{this web page}.}
\item This in turns leads to a new temperature change:
\be
\Delta T_1 = -F_1/\lambda_0 = -\frac{\lambda}{\lambda_0}\, \Delta T_0 = \frac{\lambda}{\lambda_0^2}\,F.\ee 
\item This leads to a new forcing $F_2=\lambda \Delta T_1 = (\lambda/\lambda_0)^2 F$ and a new temperature response
\be
\Delta T_2 = -F_2/\lambda_0 = -(\lambda^2/\lambda_0^3) F
\ee
\item so the total change in temperature is
\be
\Delta T_0 + \Delta T_1 + \Delta T_2 + \ldots = -\frac{F}{\lambda_0}\, \left[ 1 - \frac{\lambda}{\lambda_0} + \left( \frac{\lambda}{\lambda_0} \right)^2 + \ldots \right].
\ee
\item The sum is equal to $1/(1+x)$, so the total change in temperature is
\be
\Delta T = -\frac{F}{\lambda_0 + \lambda}.\ee
\item Using only water vapor, which turns out to be a good estimate because the other feedbacks cancel, leads to
\be
\Delta T = -\frac{F}{1.6 W/m^2} \, K\ee
\eei

\Spng{1990}{1990 IPCC prediction}
To date, the impact from $CO_2$ increase from 280 to 420 ppm then led to a forcing of 2W/m$^2$, which should have led to a temperature increase of about 1.2K. This is almost exactly what is seen in Figure~\rf{1990}, which was predicted back in 1990.

It also predicts that when the concentration doubles, the forcing will be 3.7 W/m$^2$, leading to a temperature increase of 3.7/1.6=2.3$^\circ$C. Fig.~\rf{ipcc6fig} shows some different scenarios for when this will happen.

\Sfig{ipcc6fig}{Different scenarios}

\end{document}

