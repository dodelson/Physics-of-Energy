\documentclass[11pt]{article}
\usepackage{graphicx}
\usepackage{hyperref}
\begin{document}
\newcommand{\problem}[1]{%\addtocounter{problemc}{1}
\item {#1}
}
\newcommand{\probl}[1]{\label{#1}}
\def\be{\begin{equation}}
\def\ee{\end{equation}}
\def\bea{\begin{eqnarray}}
\def\eea{\end{eqnarray}}
\newcommand{\vs}{\nonumber\\}
\def\across{a^\times}
\def\tcross{T^\times}
\def\ccross{C^\times}
\newcommand{\ec}[1]{Eq.~(\ref{eq:#1})}
\newcommand{\eec}[2]{Eqs.~(\ref{eq:#1}) and (\ref{eq:#2})}
\newcommand{\Ec}[1]{(\ref{eq:#1})}
\newcommand{\eql}[1]{\label{eq:#1}}
\newcommand{\sfig}[2]{
\includegraphics[width=#2]{#1}
        }
\newcommand{\sfigr}[2]{
\includegraphics[angle=270,origin=c,width=#2]{#1}
        }
\newcommand{\sfigra}[2]{
\includegraphics[angle=90,origin=c,width=#2]{#1}
        }
\newcommand{\Sfig}[2]{
   \begin{figure}[thbp]
   \begin{center}
    \sfig{#1.pdf}{0.5\columnwidth}
    \caption{{\small #2}}
    \label{fig:#1}
     \end{center}
   \end{figure}
}

\newcommand\dirac{\delta_D}
\newcommand{\rf}[1]{\ref{fig:#1}}
\newcommand\rhoc{\rho_{\rm cr}}
\newcommand\zs{D_S}
\newcommand\dts{\Delta t_{\rm Sh}}
\newcommand\zle{D_L}
\newcommand\zsl{D_{SL}}
\newcommand\sh{\gamma}
\newcommand\surb{\mathcal{S}}
\newcommand\psf{\mathcal{P}}
\newcommand\spsf{\sigma_{\rm PSF}}
\newcommand\bei{\begin{itemize}}
\newcommand\eei{\end{itemize}}
\begin{centering}
{\bf Homework 4: Due Thursday, February 15, 2024}
\end{centering}

\begin{enumerate}

%\problem{Problem 2.1 in the book}
%\problem{A hot air balloon has a volume of 3000 m$^3$ and carries a mass (including basket, people, etc.) of 600 kg at a height of 900 meters above the surface of the Earth. Atmospheric pressure depends on altitude via
%\be
%P(z) = P_0 e^{-z/H}\ee
%with $P_0$ equal to 1 atmosphere and $H=8.5$km, and assume that the pressure inside the balloon is equal to that outside. Assume that the atmospheric temperature is 20$^\circ$C with zero lapse rate. What is the temperature of the air inside the balloon?}
\problem{Last week, you showed that 
\be
\frac{P}{g} =\int_0^\infty dz \rho(z) \ee
where$P$ is the pressure on the surface of the Earth, $z=0$ corresponds to the surface of the Earth, and infinity is way out in the atmosphere. What is the dimension of right hand side? Assume that both sides of this equation are the same everywhere on the surface of the Earth and integrate over the surface area of the Earth. What is the dimension of the right hand side after this integral and what does $\int dA\int dz \rho$ correspond to? [Note: You do not have to do the $z$ integral.] Given that, assume that the average mass of a molecule in the atmosphere is 29g/mole ($4.8\times 10^{-26}$ kg) and the value of the left hand side (taking $P=10^5$ Pascal), obtain the number of molecules in the atmosphere.}
\problem{Problem 6.6}
\problem{Problem 6.15}
\problem{Problem 6.16}
\problem{Accessing the data set we have been using, plot on the same graph the measured $CO_2$ abundance in the atmosphere in GtC (as you did for HW3) {\it and} the amount predicted. To get the amount predicted. start with the 1958 value, and add in 1/2 of the emissions from the world every year (using your code from HW1).}

%make a more refined estimate of how much $CO_2$ every country will emit for each of the next 50 years. For example, a simple thing to do would be to find the slope of emissions (change in emissions per year) over the past 10 years say and use that to extrapolate forwards. Slightly more complex would be to use this estimate for half of the countries with low emissions per capita (``under-developed'' countries) on the assumption that their emissions will continue to go up and keep the other half of the countries flat. Or try something more ambitious.}

\end{enumerate}

\end{document}