\documentclass[11pt]{article}
\usepackage{graphicx}
\usepackage{hyperref}
\begin{document}
\newcommand{\problem}[1]{%\addtocounter{problemc}{1}
\item {#1}
}
\newcommand{\probl}[1]{\label{#1}}
\def\be{\begin{equation}}
\def\ee{\end{equation}}
\def\bea{\begin{eqnarray}}
\def\eea{\end{eqnarray}}
\newcommand{\vs}{\nonumber\\}
\def\across{a^\times}
\def\tcross{T^\times}
\def\ccross{C^\times}
\newcommand{\ec}[1]{Eq.~(\ref{eq:#1})}
\newcommand{\eec}[2]{Eqs.~(\ref{eq:#1}) and (\ref{eq:#2})}
\newcommand{\Ec}[1]{(\ref{eq:#1})}
\newcommand{\eql}[1]{\label{eq:#1}}
\newcommand{\sfig}[2]{
\includegraphics[width=#2]{#1}
        }
\newcommand{\sfigr}[2]{
\includegraphics[angle=270,origin=c,width=#2]{#1}
        }
\newcommand{\sfigra}[2]{
\includegraphics[angle=90,origin=c,width=#2]{#1}
        }
\newcommand{\Sfig}[2]{
   \begin{figure}[thbp]
   \begin{center}
    \sfig{#1.pdf}{0.5\columnwidth}
    \caption{{\small #2}}
    \label{fig:#1}
     \end{center}
   \end{figure}
}

\newcommand\dirac{\delta_D}
\newcommand{\rf}[1]{\ref{fig:#1}}
\newcommand\rhoc{\rho_{\rm cr}}
\newcommand\zs{D_S}
\newcommand\dts{\Delta t_{\rm Sh}}
\newcommand\zle{D_L}
\newcommand\zsl{D_{SL}}
\newcommand\sh{\gamma}
\newcommand\surb{\mathcal{S}}
\newcommand\psf{\mathcal{P}}
\newcommand\spsf{\sigma_{\rm PSF}}
\newcommand\bei{\begin{itemize}}
\newcommand\eei{\end{itemize}}
\begin{centering}
{\bf Homework 6: Due Thursday, March 24, 2022}
\end{centering}

\begin{enumerate}

%\problem{Problem 2.1 in the book}
\problem{Problem 19.2}
\problem{Problem 19.3}
\problem{Problem 22.1}
\problem{Problem 23.5, requires reading section 23.4}
\problem{Upload what you have of your project to date. The fiducial project is an essay that relates the current energy use to global warming, and extrapolates by how much the temperature will rise in the future. So far, you should be able to fully write 2/3 of this argument: (i) how much energy we are using, are projected to use, and various ways to reduce that; (ii) how much $CO_2$ is emitted per J of energy used and how much of that remains in the atmosphere, and again ways that we might reduce that.}

\end{enumerate}

\end{document}