\documentclass[11pt]{article}
\usepackage{graphicx}
\usepackage{hyperref}
\usepackage{mathabx}
\begin{document}
\newcommand{\problem}[1]{%\addtocounter{problemc}{1}
\newpage
Problem\ {#1}
}
\newcommand{\probl}[1]{\label{#1}}
\def\be{\begin{equation}}
\def\ee{\end{equation}}
\def\bea{\begin{eqnarray}}
\def\eea{\end{eqnarray}}
\newcommand{\vs}{\nonumber\\}
\def\across{a^\times}
\def\tcross{T^\times}
\def\ccross{C^\times}
\newcommand{\ec}[1]{Eq.~(\ref{eq:#1})}
\newcommand{\eec}[2]{Eqs.~(\ref{eq:#1}) and (\ref{eq:#2})}
\newcommand{\Ec}[1]{(\ref{eq:#1})}
\newcommand{\eql}[1]{\label{eq:#1}}
\newcommand{\sfig}[2]{
\includegraphics[width=#2]{#1}
        }
\newcommand{\sfigr}[2]{
\includegraphics[angle=270,origin=c,width=#2]{#1}
        }
\newcommand{\sfigra}[2]{
\includegraphics[angle=90,origin=c,width=#2]{#1}
        }
\newcommand{\Sfig}[2]{
   \begin{figure}[thbp]
   \begin{center}
    \sfig{../Figures/#1.pdf}{0.5\columnwidth}
    \caption{{\small #2}}
    \label{fig:#1}
     \end{center}
   \end{figure}
}
\newcommand{\Sjpeg}[2]{
   \begin{figure}[thbp]
   \begin{center}
    \sfig{../Figures/#1.jpeg}{0.7\columnwidth}
    \caption{{\small #2}}
    \label{fig:#1}
     \end{center}
   \end{figure}
}
\newcommand{\Spng}[2]{
   \begin{figure}[thbp]
   \begin{center}
    \sfig{../Figures/#1.png}{\columnwidth}
    \caption{{\small #2}}
    \label{fig:#1}
     \end{center}
   \end{figure}
}


\newcommand\dirac{\delta_D}
\newcommand{\rf}[1]{\ref{fig:#1}}
\newcommand\rhoc{\rho_{\rm cr}}
\newcommand\zs{D_S}
\newcommand\dts{\Delta t_{\rm Sh}}
\newcommand\zle{D_L}
\newcommand\zsl{D_{SL}}
\newcommand\sh{\gamma}
\newcommand\surb{\mathcal{S}}
\newcommand\psf{\mathcal{P}}
\newcommand\spsf{\sigma_{\rm PSF}}
\newcommand\bei{\begin{itemize}}
\newcommand\eei{\end{itemize}}
\newcommand\bee{\begin{enumerate}}
\newcommand\eee{\end{enumerate}}

\renewcommand{\theenumi}{\Alph{enumi}}
{\bf Name:}
\bigskip
\bigskip
\bigskip
\bigskip
\bigskip

\begin{centering}
{\bf Midterm, Physics of Energy 33-226}
\end{centering}
There are 5 problems in total. Each of the 5 problems is worth the same number of points.

%\begin{enumerate}

%\problem{Problem 2.1 in the book}

\problem{1. \Sfig{sh1}{Heat capacity in units of $Nk_B$ for a given diatomic molecule.}
\bee
\item Why is the value at very low temperature equal to 1.5?
\bigskip
\bigskip
\bigskip
\bigskip
\bigskip
\bigskip
\item Why does it rise to 2.5 at higher temperatures?
\bigskip
\bigskip
\bigskip
\bigskip
\bigskip
\bigskip
\item Why does it rise to 3.5 at the highest temperature?
\bigskip
\bigskip
\bigskip
\bigskip
\bigskip
\bigskip
\item From this graph, estimate the value of the first excited rotational state of the molecule (in J).
\bigskip
\bigskip
\bigskip
\bigskip
\bigskip
\bigskip
\bigskip
\bigskip
\bigskip
\bigskip
\bigskip
\bigskip
\item Estimate the ground state vibrational energy (in J).
\eee}

\problem{2. In class, we found that equating the radiation coming in from the Sun and the radiation leaving the Earth led -- under simplifying conditions -- to a temperature on Earth of 291K. That calculation assumed that the Earth absorbed {\it all} of the incoming radiation.
How would the temperature change if we accounted for the fact that ice on Earth reflects back 10\% of the incoming solar radiation?
}


\problem{3. Suppose that the temperature of the water in the ocean at a particular place is 30$^\circ$C from the surface down to 100m, and then below that the temperature decreases at a constant rate until it reaches 5$^\circ$C at a depth of 1000m. The thermal conductivity of seawater is $k=0.60$W/m/K; its heat capacity is $C=4$kJ/kg/K; and its density is $10^3$kg/m$^3$.
\bee
\item What is the heat flux downwards between 100 and 1000m? 
\bigskip
\bigskip
\bigskip
\bigskip
\bigskip
\bigskip
\bigskip
\bigskip
\bigskip
\bigskip
\bigskip
\bigskip

\item All water below 1000m is at the same temperature for a variety of reasons not of interest here. At the rate of heat flux calculated in A, how long would it take to heat a column of water in the ocean from 1000m down to 2000m by 5$^\circ$C? 
\eee}

\problem{4. Fig.~\rf{co2ppm} shows that the amount of $CO_2$ in the atmosphere in 2020 was about 420 ppm, so that for every million molecules in the atmosphere, there were 420 $CO_2$ molecules. Take the density of molecules in the atmosphere to depend only on the distance from the surface of the Earth, $z$:
\be
n(z) = n_0 e^{-z/H}\ee
where $H=8.5$km, and $n_0=2.5\times 10^{25}$ air molecules per $m^3$.
\Sjpeg{co2ppm}{Fraction of $CO_2$ in the atmosphere as a function of time. The units are parts per million: the number of $CO_2$ molecules as a fraction of all the molecules in the atmosphere.}
\bee
%\item What is the mass of one $CO_2$ molecule in kg?
%\bigskip
%\bigskip
%\bigskip
%\bigskip
%\bigskip
%\bigskip
%\bigskip
%\bigskip
%\bigskip
%\bigskip
%\bigskip
%\bigskip
\item How many $CO_2$ molecules are in 1 GtC (Gigatonnes of carbon)?
\bigskip
\bigskip
\bigskip
\bigskip
\bigskip
\bigskip
\bigskip
\bigskip
\bigskip
\bigskip
\bigskip
\bigskip
\item What is the total number of air molecules in the atmosphere?
\bigskip
\bigskip
\bigskip
\bigskip
\bigskip
\bigskip
\bigskip
\bigskip
\bigskip
\bigskip
\bigskip
\bigskip
\item Use your answer from A and B to convert 420 ppm into GtC .
\eee
}


\problem{5. We have several times assumed that the pressure in the atmosphere $P=P_0e^{-z/H}$ with $H=8.5$km. Use this and also assume the atmosphere is in hydrostatic equilibrium (\ec{hydro}) and is an ideal gas with constant composition of 80\% nitrogen ($N_2$) and 20\% oxygen ($O_2$). Show that -- under these assumptions -- the temperature is constant with height. What is the temperature?}




%\end{enumerate}
\newpage
\appendix
\section{Formula Sheet}
\subsection{Fundamental Constants}
\bei
\item 1 Gigatonne = $10^9$ tonnes = $10^9\times 10^3$ kg
\item Speed of Light\be
c=3\times 10^8 m/s\ee
\item Boltzmann Constant\be
k_B=1.38\times 10^{-23} J/K \ee
\item Reduced\ Planck's constant
\be
\hbar\equiv \frac{h}{2\pi} = 1.05\times 10^{-34} J-s
\ee
\item Stefan-Boltzmann\, constant
\be
\sigma = 5.67\times 10^{-8} W/m^2/K^4\ee
\item Proton Mass
\be m_p=1.67\times 10^{-27} kg\ee
\item Common elements (if asked for their mass, just multiply the number of nucleons by the mass of the proton): Carbon (12 nucleons); Oxygen (16); Nitrogen (14); Iron (56)
\item Radius of Earth, Sun
\be R_\Earth = 6.4\times 10^6 m.
\ee
\be R_\odot = 7\times 10^8 m\ee
\eei
\subsection{Equations}
\bei
\item Properties of Photons
\be
E = \hbar\omega = h\nu = h\frac{c}{\lambda} .\ee
\item Ideal Gas Law
\be PV = Nk_BT\ee
\item Pressure = Force Per Area
\item First Law of Thermodynamics
\be dQ = dU + PdV
\ee
\item Internal Energy, $U$: Every degree of freedom that is accessible at a given temperature contributes $k_BT/2$ to $U$.
\item Rotational Energy Levels ($I$ is the moment of inertia)
\be E_{R,L} = E_R \frac{L(L+1)}{2I}\qquad L=0,1,2,\ldots\ee
\item Vibrational Energy Levels ($\omega$ is the frequency)
\be
E_{v,n} = {\hbar\omega}\left[ n + \frac12\right].\qquad n=0,1,2,\ldots\ee
\item Heat Capacities:
\bea
C_V &=& \frac{dQ}{dT}\Big\vert_V = \frac{\partial U}{\partial T}\Big\vert_V \Rightarrow \Delta Q = C_V \Delta T\vs
C_P &=& \frac{dQ}{dT}\Big\vert_P = \frac{\partial U}{\partial T}\Big\vert_P + P \frac{\partial V}{\partial T}\Big\vert_P
\eea
\item Specific Heat Capacity $c=C/M$ where $M$ is the total mass being heated.
\item Fourier's Law
\be
\vec q(\vec x) = -k \nabla T(\vec x)\ee
\item Conservation Law for heat flow
\be
\frac{\partial u(\vec x,t)}{\partial T} = -\nabla\cdot \vec q(\vec x).\ee
\item Hydrostatic Equilibrium
\be
\frac{dP}{dz} = -\rho g.\eql{hydro}\ee
\item Thermal Resistance
\be R_{\rm thermal} = \frac{L}{k}.\ee
\item Stefan-Boltzmann Law for Emission of Thermal Radiation
\be \frac{dQ}{dt} = \sigma T^4 A\ee
\item Spectrum of Thermal Radiation (with dimensions of Energy per time per area per frequency)
\be
\frac{dP}{dAd\omega} = \frac{\hbar\omega^3}{4\pi^2c^2}\,\frac1{e^{\hbar\omega/k_BT}-1}.\ee
\eei
%\Spng{periodic}{Periodic Table: You can assume that each element has mass equal to its atomic weight (without the decimal points) times the mass of the proton.}
\end{document}