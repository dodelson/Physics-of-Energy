\documentclass[11pt]{article}
\usepackage{graphicx}
\usepackage{hyperref}
\usepackage{mathabx}
\begin{document}
\newcommand{\problem}[1]{%\addtocounter{problemc}{1}
\newpage
Problem\ {#1}
}
\newcommand{\probl}[1]{\label{#1}}
\def\be{\begin{equation}}
\def\ee{\end{equation}}
\def\bea{\begin{eqnarray}}
\def\eea{\end{eqnarray}}
\newcommand{\vs}{\nonumber\\}
\def\across{a^\times}
\def\tcross{T^\times}
\def\ccross{C^\times}
\newcommand{\ec}[1]{Eq.~(\ref{eq:#1})}
\newcommand{\eec}[2]{Eqs.~(\ref{eq:#1}) and (\ref{eq:#2})}
\newcommand{\Ec}[1]{(\ref{eq:#1})}
\newcommand{\eql}[1]{\label{eq:#1}}
\newcommand{\sfig}[2]{
\includegraphics[width=#2]{#1}
        }
\newcommand{\sfigr}[2]{
\includegraphics[angle=270,origin=c,width=#2]{#1}
        }
\newcommand{\sfigra}[2]{
\includegraphics[angle=90,origin=c,width=#2]{#1}
        }
\newcommand{\Sfig}[2]{
   \begin{figure}[thbp]
   \begin{center}
    \sfig{../Figures/#1.pdf}{0.5\columnwidth}
    \caption{{\small #2}}
    \label{fig:#1}
     \end{center}
   \end{figure}
}
\newcommand{\Sjpeg}[2]{
   \begin{figure}[thbp]
   \begin{center}
    \sfig{../Figures/#1.jpeg}{0.7\columnwidth}
    \caption{{\small #2}}
    \label{fig:#1}
     \end{center}
   \end{figure}
}
\newcommand{\Spng}[2]{
   \begin{figure}[thbp]
   \begin{center}
    \sfig{../Figures/#1.png}{\columnwidth}
    \caption{{\small #2}}
    \label{fig:#1}
     \end{center}
   \end{figure}
}


\newcommand\dirac{\delta_D}
\newcommand{\rf}[1]{\ref{fig:#1}}
\newcommand\rhoc{\rho_{\rm cr}}
\newcommand\zs{D_S}
\newcommand\dts{\Delta t_{\rm Sh}}
\newcommand\zle{D_L}
\newcommand\zsl{D_{SL}}
\newcommand\sh{\gamma}
\newcommand\surb{\mathcal{S}}
\newcommand\psf{\mathcal{P}}
\newcommand\spsf{\sigma_{\rm PSF}}
\newcommand\bei{\begin{itemize}}
\newcommand\eei{\end{itemize}}
\newcommand\bee{\begin{enumerate}}
\newcommand\eee{\end{enumerate}}

\renewcommand{\theenumi}{\Alph{enumi}}
{\bf Name:}
\bigskip
\bigskip
\bigskip
\bigskip
\bigskip

\begin{centering}
{\bf Midterm, Physics of Energy 33-226}
\end{centering}
There are 5 problems in total. Each of the 5 problems is worth the same number of points.

%\begin{enumerate}

%\problem{Problem 2.1 in the book}

\problem{1. A dishwasher completes a cycle in one hour using 2kW of power. 
\bee
\item How much energy is used in the cycle (in J)?
\bigskip
\bigskip
\bigskip
\bigskip
\bigskip
\bigskip
\bigskip
\bigskip
\bigskip
\bigskip
\bigskip
\bigskip
\item Water flows from a faucet at 2 gallons\footnote{1 Gallon is $0.00379m^3$.} per minute. If you wash dishes yourself using hot water (the same as the shower in the problem in class $T=40^\circ$C vs. room temperature of $10^\circ$) from the faucet in the sink, how much energy do you use if you wash dishes for 15 minutes? 
\eee
}

\problem{2. Science magazine printed a story recently about an energy storage device in Tennessee's Raccoon Mountain, where water is pumped to an upper reservoir and then -- when other power sources are low -- water falls down and turbines that generate electricity.
\bee
\item Assume that the water is pumped up by a 330m shaft and that $4\times 10^{10}$ liters of water is pumped to the upper reservoir. What is the gain in potential energy?  \bigskip
\bigskip
\bigskip
\bigskip
\bigskip
\bigskip
\bigskip
\bigskip
\bigskip
\bigskip
\bigskip
\bigskip
\item To give an indication of how fast the water is pumped, the article says ``The pumps draw water from the Tennessee [River] and shoot it straight up the 10-meter-wide shaft at a rate that would fill an Olympic pool in less than 6 seconds.'' An Olympic pool holds 2.5 million liters of water. How much power is required to shoot the water up?
\bigskip
\bigskip
\bigskip
\bigskip
\bigskip
\bigskip
\bigskip
\bigskip
\bigskip
\bigskip
\bigskip
\item The article goes on to say that when needed the water drains down the shaft and spins the turbines ``generating 1700 megawatts of electricity.'' Given the energy you computed in A, for how many hours can it deliver this much power?
\eee}
\problem{3. The
vibrational-rotational energy levels of a diatomic
molecule are given by 
\be E(n, J) = \hbar\omega_{\rm vib} (n+1/2)+\epsilon^{\rm rot} J(J+
1).\ee
 Here $\hbar\omega_{\rm vib}=0.266$ eV, $\epsilon^{\rm rot}=2.4\times10^{-4}$ eV and $J = 0, 1, 2, \ldots$  is
the rotational quantum number. 
. 
\bee
\item
Ignoring rotations, what is the wavelength
$\lambda_{\rm vib}$  (in microns) of the radiation absorbed when the
CO molecule makes a transition from vibrational level n to n + 1? 
\bigskip
\bigskip
\bigskip
\bigskip
\bigskip
\bigskip
\bigskip
\bigskip
\bigskip
\bigskip
\bigskip
\bigskip

\item Vibrational transitions are always accompanied
by a transition from rotational level J to $J\pm 1$. The
result is a sequence of equally spaced absorption lines
centered at $\lambda_{\rm vib}$ . What is the spacing between these
vibrational-rotational absorption lines (in microns)?
\eee
}

\problem{4. When (under what conditions) does hot air rise? Why?}
%\problem{3. Suppose that the temperature of the water in the ocean at a particular place is 30$^\circ$C from the surface down to 100m, and then below that the temperature decreases at a constant rate until it reaches 5$^\circ$C at a depth of 1000m. The thermal conductivity of seawater is $k=0.60$W/m/K; its heat capacity is $C=4$kJ/kg/K; and its density is $10^3$kg/m$^3$.
%\bee
%\item What is the heat flux downwards between 100 and 1000m? 
%\bigskip
%\bigskip
%\bigskip
%\bigskip
%\bigskip
%\bigskip
%\bigskip
%\bigskip
%\bigskip
%\bigskip
%\bigskip
%\bigskip
%
%\item All water below 1000m is at the same temperature for a variety of reasons not of interest here. At the rate of heat flux calculated in A, how long would it take to heat a column of water in the ocean from 1000m down to 2000m by 5$^\circ$C? 
%\eee}

%\problem{4. Fig.~\rf{co2ppm} shows that the amount of $CO_2$ in the atmosphere in 2020 was about 420 ppm, so that for every million molecules in the atmosphere, there were 420 $CO_2$ molecules. Take the density of molecules in the atmosphere to depend only on the distance from the surface of the Earth, $z$:
%\be
%n(z) = n_0 e^{-z/H}\ee
%where $H=8.5$km, and $n_0=2.5\times 10^{25}$ air molecules per $m^3$.
%%\Sjpeg{co2ppm}{Fraction of $CO_2$ in the atmosphere as a function of time. The units are parts per million: the number of $CO_2$ molecules as a fraction of all the molecules in the atmosphere.}
%\bee
%%\item What is the mass of one $CO_2$ molecule in kg?
%%\bigskip
%%\bigskip
%%\bigskip
%%\bigskip
%%\bigskip
%%\bigskip
%%\bigskip
%%\bigskip
%%\bigskip
%%\bigskip
%%\bigskip
%%\bigskip
%\item How many $CO_2$ molecules are in 1 GtC (Gigatonnes of carbon)?
%\bigskip
%\bigskip
%\bigskip
%\bigskip
%\bigskip
%\bigskip
%\bigskip
%\bigskip
%\bigskip
%\bigskip
%\bigskip
%\bigskip
%\item What is the total number of air molecules in the atmosphere?
%\bigskip
%\bigskip
%\bigskip
%\bigskip
%\bigskip
%\bigskip
%\bigskip
%\bigskip
%\bigskip
%\bigskip
%\bigskip
%\bigskip
%\item Use your answer from A and B to convert 420 ppm into GtC .
%\eee
%}
%

\problem{5. We have several times assumed that the pressure in the atmosphere $P=P_0e^{-z/H}$ with $H=8.5$km. Use this and also assume the atmosphere is in hydrostatic equilibrium (\ec{hydro}) and is an ideal gas with constant composition of 80\% nitrogen ($N_2$) and 20\% oxygen ($O_2$). Show that -- under these assumptions -- the temperature is constant with height. What is the temperature?}




%\end{enumerate}
\newpage
\appendix
\section{Formula Sheet}
\subsection{Fundamental Constants and Units}
\bei
\item 1 Gigatonne = $10^9$ tonnes = $10^9\times 10^3$ kg
\item 1 Liter $= 10^{-3} m^3$
\item Speed of Light\be
c=3\times 10^8 m/s\ee
\item Boltzmann Constant\be
k_B=1.38\times 10^{-23} J/K \ee
\item electron volts
\be 1 eV = 1.6\times 10^{-19} J.\ee
\item Reduced\ Planck's constant
\be
\hbar\equiv \frac{h}{2\pi} = 1.05\times 10^{-34} J-s
\ee
\be
\hbar c = 2\times 10^{-7} eV-m
\ee
\item Stefan-Boltzmann\, constant
\be
\sigma = 5.67\times 10^{-8} W/m^2/K^4\ee
\item Gravitational acceleration on Earth
\be
g=9.8 m/s^2.\ee
\item Proton Mass
\be m_p=1.67\times 10^{-27} kg\ee
\item Common elements (if asked for their mass, just multiply the number of nucleons by the mass of the proton): Carbon (12 nucleons); Oxygen (16); Nitrogen (14); Iron (56)
\item Radius of Earth, Sun
\be R_\Earth = 6.4\times 10^6 m.
\ee
\be R_\odot = 7\times 10^8 m\ee
\eei
\subsection{Equations}
\bei
\item Energy lost to air resistance
\be
\Delta E = \frac12 c_d A D \rho v^2\ee
\item Properties of Photons
\be
E = \hbar\omega = h\nu = h\frac{c}{\lambda} .\ee
\item Ideal Gas Law
\be PV = Nk_BT\ee
\item Pressure = Force Per Area
\item First Law of Thermodynamics
\be dQ = dU + PdV
\ee
\item Internal Energy, $U$: Every degree of freedom that is accessible at a given temperature contributes $k_BT/2$ to $U$.
\item Rotational Energy Levels ($I$ is the moment of inertia)
\be E^{\rm rot}_{J} = \epsilon^{\rm rot}\,J(J+1)\qquad J=0,1,2,\ldots\ee
\item Vibrational Energy Levels ($\omega$ is the frequency)
\be
E^{\rm vib}_{n} = {\hbar\omega}\left[ n + \frac12\right].\qquad n=0,1,2,\ldots\ee
\item Heat Capacities:
\bea
C_V &=& \frac{dQ}{dT}\Big\vert_V = \frac{\partial U}{\partial T}\Big\vert_V \Rightarrow \Delta Q = C_V \Delta T\vs
C_P &=& \frac{dQ}{dT}\Big\vert_P = \frac{\partial U}{\partial T}\Big\vert_P + P \frac{\partial V}{\partial T}\Big\vert_P
\eea
\item Specific Heat Capacity $c=C/M$ where $M$ is the total mass being heated.
\item Fourier's Law
\be
\vec q(\vec x) = -k \nabla T(\vec x)\ee
\item Conservation Law for heat flow
\be
\frac{\partial u(\vec x,t)}{\partial t} = -\nabla\cdot \vec q(\vec x).\ee
\item Heat Equation
\be
\frac{\partial}{\partial t} T= a\nabla^2 T
\ee
with $a\equiv k/\rho c_p$. 
\item Hydrostatic Equilibrium
\be
\frac{dP}{dz} = -\rho g.\eql{hydro}\ee
\item Thermal Resistance
\be R_{\rm thermal} = \frac{L}{k}.\ee
\item Condition for Conductive Stability
\be
\vert \frac{dT}{dz} \vert < \Gamma_{\rm adiabatic} \ee
and for dry air, $\Gamma_{\rm dry\ adiabatic}= \frac{g}{c_p}.$
\item Stefan-Boltzmann Law for Emission of Thermal Radiation
\be \frac{dQ}{dt} = \sigma T^4 A\ee
\item Spectrum of Thermal Radiation (with dimensions of Energy per time per area per frequency)
\be
\frac{dP}{dAd\omega} = \frac{\hbar\omega^3}{4\pi^2c^2}\,\frac1{e^{\hbar\omega/k_BT}-1}.\ee
\eei
\subsection{Some Values}
\bei
\item Density of air at the surface of the Earth: 1.2 kg/m$^3$
\item Population of the World: 8 billion people
\item $CO_2$ measured in the atmosphere: 900 GtC 
\item Specific Heat Capacity $c_p$ (kJ/kg/K)
\bee
\item Water: 4.18
\item Ice: 2.05
\item Air: 1
\eee
\item Latent Heat of Ice to Water: $334$ kJ/kg
\item Density at standard conditions on the surface of the Earth (kg/m$^3$):
\bee 
\item Liquid Water and Ice: $10^3$ 
\item Air: 1.2 
\eee
\item Thermal Conductivity $k$ (W/m/K):
\bee
\item Air 0.026
\item Glass 1
\eee
\eei
\end{document}