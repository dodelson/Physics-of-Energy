\documentclass[11pt]{book}
\usepackage{graphicx}
\usepackage{amsmath,mathtools,mathabx}
\usepackage{hyperref}

\begin{document}
\newcommand{\problem}[1]{%\addtocounter{problemc}{1}
\item {#1}
}
\newcommand{\probl}[1]{\label{#1}}
\def\be{\begin{equation}}
\def\ee{\end{equation}}
\def\bea{\begin{eqnarray}}
\def\eea{\end{eqnarray}}
\newcommand{\vs}{\nonumber\\}
\def\across{a^\times}
\def\tcross{T^\times}
\def\ccross{C^\times}
\newcommand{\ec}[1]{Eq.~(\ref{eq:#1})}
\newcommand{\eec}[2]{Eqs.~(\ref{eq:#1}) and (\ref{eq:#2})}
\newcommand{\Ec}[1]{(\ref{eq:#1})}
\newcommand{\eql}[1]{\label{eq:#1}}
\newcommand{\sfig}[2]{
\includegraphics[width=#2]{#1}
        }
\newcommand{\sfigr}[2]{
\includegraphics[angle=270,origin=c,width=#2]{#1}
        }
\newcommand{\sfigra}[2]{
\includegraphics[angle=90,origin=c,width=#2]{#1}
        }
\newcommand{\Sfig}[2]{
   \begin{figure}[thbp]
   \begin{center}
    \sfig{Figures/#1.pdf}{0.7\columnwidth}
    \caption{{\small #2}}
    \label{fig:#1}
     \end{center}
   \end{figure}
}
\newcommand{\Sfigl}[2]{
   \begin{figure}[thbp]
   \begin{center}
    \sfig{Figures/#1.pdf}{0.9\columnwidth}
    \caption{{\small #2}}
    \label{fig:#1}
     \end{center}
   \end{figure}
}
\
\newcommand{\Sjpg}[2]{
   \begin{figure}[thbp]
   \begin{center}
    \sfig{../Figures/#1.jpg}{0.8\columnwidth}
    \caption{{\small #2}}
    \label{fig:#1}
     \end{center}
   \end{figure}
}
\newcommand{\Spng}[2]{
   \begin{figure}[thbp]
   \begin{center}
    \sfig{Figures/#1.png}{0.8\columnwidth}
    \caption{{\small #2}}
    \label{fig:#1}
     \end{center}
   \end{figure}
}

\newcommand{\Sfigr}[2]{
   \begin{figure}[thbp]
   \begin{center}
    \sfigr{../Figures/#1.pdf}{0.5\columnwidth}
    \caption{{\small #2}}
    \label{fig:#1}
     \end{center}
   \end{figure}
}

\newcommand\dirac{\delta_D}
\newcommand{\rf}[1]{\ref{fig:#1}}
\newcommand\example[1]{{\tt EXAMPLE: #1}}
\newcommand\expect[1]{{\tt {\bf Back of the Envelope:} #1}}
\newcommand\theorem[1]{{\tt Theorem: #1}}
\newcommand\bei{\begin{itemize}}
\newcommand\eei{\end{itemize}}
\newcommand\bee{\begin{enumerate}}
\newcommand\eee{\end{enumerate}}
\newcommand\lecture[1]{\newpage
\begin{center}
Lecture #1
\end{center}
}
\newcommand\conversion[1]{\fbox{#1}}

\chapter{Overview}
\lecture{1}

\bei
\item Annual Per Capita Energy Use is 73 GJ ($10^9$J)
\item
Annual World Energy Use in EJ is 562.1 EJ ($10^{18}$J)
\item
Global Power Use is 17.82 TW ($10^{12}$W)
\item 
Global Energy Consumption is then 156138 TW-hours $=562$ EJ
\item
Daily Energy Use per capita is 200 MJ
\eei
\chapter{Mechanical Energy}

Analyze the energetics of a car trip from Boston to New York. Since the distance is 210 miles and considering a car that gets 30 miles per gallon, the trip requires 7 gallons of gas. 

\conversion{1 Gallon = 3.78 Liters = 3780 cm$^3$}

The density of gas is  0.85 g/cm$^3$. So a gallon of gas has about 3200 gms. The molecule $C_6H_{14}$ has a molar mass of about $6*12+14=86$ gm/mol. So, there is about 37 moles in a gallon of gas. If we estimate the energy per mol as 100 kcal/mol, we get 3700kcal/gallon = 15 MJ/gallon, which is too low by a factor of 8. Perhaps because there are 6 carbon atoms in the molecule, the binding energy is about 800 kcal/mol. The total energy per gallon of gas is 120MJ.

Seven gallons of gas therefore contain 840MJ of energy. However, transforming that energy into usable energy is not 100\% efficient. It is roughly 25\% efficient, so buying 7 gallons of gas gives you about 210MJ of energy. How is that used to get from Boston to New York?

\bei
\item {\bf Kinetic Energy} First calculate the difference between the kinetic energy of the car moving at 60mph and the car at rest (0). The energy in the gas must be used to give the car this kinetic energy, equal to 0.65 MJ. This is assuming no stops. Even assuming 20 stops, though, leads to only {\bf 13 MJ}, far smaller than the 210 we are trying to account for. 
\item {\bf Potential Energy:} Going up a hill, the car gains potential energy $mgh$ and then loses it going down. The energy per mile gained and lost is 270kJ assuming an average hill height of 15m per mile. To ensure not going too slow or too fast, though, the driver will either have to provide more gas or more break to keep a stable speed. A rough estimate is that half of that 270kJ gets consumed, so the total energy used over the 210 miles is roughly {\bf 28 MJ}.
\item {\bf Air Resistance:} Toy model has each air molecule picking up the velocity of the car, so the total kinetic energy transferred to the air (and therefore lost by the car) equal to $dMv^2/2$ where $dM$ is the mass in an infinitesmal volume $dV = Adx$ where $A$ is the cross-sectional area of the car. So, after traveling a total distance $D$, the energy lost to air friction is
\be
\Delta E \sim \frac12 \rho A D v^2
.\ee 
The twiddle here reflects the fact that this is an approximation, rectified by inserting a fudge factor, $c_d$, the drag coefficient, which is of order 1/3 for cars. Putting in $c_d$, and taking $A=2.7$m$^2; \rho=1.2$kg/m$^3$ (for air) leads to an energy loss of {\bf 130 MJ}.
\item {\bf Friction:} The energy lost to friction of the tires with the road is estimated in the text to be {\bf 50 MJ}.
\eei
The total estimate is therefore about 220 MJ, which is roughly correct.

\chapter{Heat and Thermal Energy}
\section{Equipartition Theorem}
The kinetic energy of a system of $N$ particles is equal to
\be
\langle E \rangle = \int d^3v f(\vec v)\,\frac{mv^2}{2}
\ee
Here $f(v)$ is the velocity distribution normalized so that
\be
N = \int d^3v f(v)
.\ee
In equilibrium, $f\propto e^{-E/k_BT} = e^{-mv^2/2k_BT}$, so normalization leads to
\bea
N &=& C4\pi\int_0^\infty dv v^2 e^{-mv^2/2k_BT}
\vs 
&=& C4\pi \left(\frac{2k_BT}{m} \right)^{3/2}\int_0^\infty dx x^2 e^{-x^2}
\vs
&=&
C4\pi \left(\frac{2k_BT}{m} \right)^{3/2}\,\frac{\sqrt{\pi}}4
\vs
&=&
C \left(\frac{2\pi k_BT}{m} \right)^{3/2}.
\eea
So,
\be
f=N \left( \frac{m}{2\pi k_BT}\right)^{3/2}\, e^{-mv^2/2k_BT}.\ee

We can now calculate the kinetic energy:
\bea
\langle E \rangle &=& N \left( \frac{m}{2\pi k_BT}\right)^{3/2}\,  \int d^3v e^{-mv^2/2k_BT}\,\frac{mv^2}{2}
\vs &=& 2\pi Nm \left( \frac{m}{2\pi k_BT}\right)^{3/2}\, \int_0^\infty dv v^4 e^{-mv^2/2k_BT} \vs
&=&
2\pi Nm \left( \frac{m}{2\pi k_BT}\right)^{3/2}\, \left( \frac{2k_BT}{m}\right)^{5/2}\,\int_0^\infty dx x^4 e^{-x^2} 
\vs
&=&
2\pi Nm \left( \frac{m}{2\pi k_BT}\right)^{3/2}\, \left( \frac{2k_BT}{m}\right)^{5/2}\, \frac{3\sqrt{\pi}}{8}
\vs
&=&
\frac{3Nk_BT}{2}.
\eea
If we had done this in 1D, we would have gotten $k_BT/2$. In general, the variance of any one component of the velocity is $k_BT/m$, so in an isotropic equilibrium situation, there is a factor of 3. 

The situation is more complex for more complex particles, starting with diatomic atoms. There, the book says you add $k_BT/2$ for each degree of freedom. Similarly for solids, the idea is there is no random kinetic energy but rather the $N$ atoms can each oscillate in one of 3 dimensions and each of these modes leads to $k_BT$ because there is a kinetic and potential term. It is really a bit more complicated: the energy of a harmonic oscillator is
\be
E_n=\hbar \omega\left(n+\frac12 \right).\qquad n=0,1,2,\ldots\ee
Again, the probability of a single one of these oscillators begin in the $n$th state is proportional to $e^{-E_n/k_BT}$. Now the expected energy is
\be
\langle E\rangle
= N\frac{\sum_n E_n e^{-E_n/k_BT}}{\sum_n e^{-E_n/k_BT}}
\ee
or
\be
\langle E\rangle = -N\frac{\partial}{\partial\beta} \,\ln Z
\ee
where $\beta\equiv 1/k_BT$ and
\be
Z\equiv \sum_ne^{-E_n/k_BT}.\ee

The low temperature limit of this $\hbar\omega \gg k_BT$ means that only the $n=0$ term will contribute and $Z=e^{-E_0\beta}$, leading to
\be
\langle E\rangle = NE_0\ee
all the particles in the ground state. The other limit is the high temperature limit, in which case we can approximate the energy as $\hbar\omega n$ and write the sum as
\be
Z= \frac{1}{1-e^{-\hbar\omega\beta}}\simeq \frac{1}{\hbar\omega\beta}\ee
Taking the derivative leads to
\be
\langle E\rangle_{hi\, T} =Nk_BT\ee
The book explains this as the harmonic oscillator having kinetic and potential energy, each of which contributes $k_BT/2$.

You can go through the same exercise for any degrees of freedom: rotational or the energy levels in an atom. You do not always get $k_BT/2$ for each degree of freedom, but the probability of finding a particle in a given energy state is always proportional to $e^{-E/k_BT}$ when the system is in equilibrium.



\section{Pressure and Efficiency}

Thermal energy can be converted into mechanical energy.  E.g., thermal motion from a gas exerts a force on a piston. In one dimension,
\bei
\item Each particle that hits the wall has its energy remain the same (so $|v|$ is constant) but it changes direction. Therefore, the change in momentum in a given bounce off the wall is $2mv$. 
\item The particle will then move in the other direction towards the fixed wall in a time $l/v$, and then return to the piston in the same time. So in a total time $2l/v$, the momentum that the particle gives to the piston is $2mv$. 
\item The force is then 
\be F = \frac{\Delta p}{\Delta t} = \frac{mv^2}{l}
\ee
\item The total force from $N$ particles drawn from a thermal distribution is
\be
F = \frac{Nm\langle v^2\rangle}{l} = \frac{Nk_BT}{l}
\ee
\item The pressure is the force per area, so
\be
PV = Nk_BT.
\ee
\eei
This is the ideal gas law.

If the region outside the piston is a vacuum such that its pressure is zero, then the work done on the piston pushing it a distance $dx$ is
\be
dW = Fdx=PAdx.\ee
On the other hand, in the more realistic case that the region outside has a pressure smaller by only a smaller amount $\Delta P$ (if, e.g., the inner region is slightly heated), then the force is only $\Delta P A$. So, the efficiency -- the fraction of work done related to maximum possible is
\be
\eta = \frac{\Delta P}{P} = \frac{T_{in}-T_{out}}{T_{in}}
\ee
where the second equality holds in the case of an ideal gas.

The internal energy of the gas in the inner region is reduced by the work it does, by an amount $PdV$ (some of this work serves to heat the gas on the other side of the piston, not just move the piston). But, it can also ain internal energy if it is heated so the total change in its internal energy is
\be
dU = dQ - PdV.\ee

When heat is added ($dQ>0$), then temperature will rise, so that $dQ=CdT$. If the volume is fixed (the piston does not move in the above example, then all the heat will go into increasing the temperature, so the temperature should rise more than if the piston moved (and the pressure remained constant because the density went down). We distinguish therefore between two heat capacities:
\bea
C_V&\equiv& \frac{\partial U}{\partial T}\Big\vert_V
\vs
C_P&\equiv& \frac{\partial U}{\partial T}\Big\vert_P + P\frac{\partial V}{\partial T}\Big\vert_P.
\eea

We have already computed $U$ for an ideal monatomic and diatomic gas. 
\bei
\item Monatomic: $C_V = (3/2) Nk_B$
\item Diatomic (high $T$): $C_V=(7/2) N k_B$ because of the 2 vibrational and 2 rotational degrees of freedom.
\eei

\chapter{Nuclear Power}

\section{Overview}

\section{Nuclear Physics}

\section{Fission}

Reaction rates are determined by cross-sections. For the reaction $B+T\rightarrow C$, the rate at which $C$ particles are produced is equal to
\be
\frac{dN_C}{dt} = \Phi_B \sigma_{BT\rightarrow C} \ee
where $\Phi_B$ is the flux, the number of particles per unit area per time. 

Fission occurs when the decay products have lower energy than the parent particle. The total binding energy of the daughters must be greater than the binding energy of the parent. A rough estimate for the energy released is to assume the particle splits in half. Then the energy released is
\be
Q = 2B(Z/2,A/2) - B(Z,A)
\ee
with $Z$ set to the minimum energy number of protons for the nuclei with mass number $A$. This leads to Fig. 18.2 and the fact that large $A>200$ make the best candidates for fuel, emitting up to 200 MeV of energy.

Key processes for neutrons:
\bei
\item Scattering off $^{238}U$; $\sigma\sim 10$ barns
\item destruction via absorption by $^{238}U$: $\sigma\sim 1$ barn
\item fission via $^{238}U$ capture: $\sigma\sim 10^{-5}$ barns for thermal neutrons but up to one barn for $E>1$ MeV
\item fission via $^{235}U$ capture: $\sigma\sim600$ barns for thermal neutrons
\item energy loss off moderator, governed by energy loss per collision (roughly $\xi$) and cross-section
\eei



\section{Fission Reactors}

Follow the neutrons: $k$ is the number of thermal neutrons generated by the previous generation of 1 thermal neutron. If $k>1$, the reactor is supercritical; if it is less than 1, it is sub-critical. The goal is to keep $k$ very close to 1; or $\rho\equiv (k-1)/k$ close to zero.

Fundamental equation
\be
k=\eta\epsilon p f
\ee
where
\bei
\item $\eta$ is the number of fast neutrons the initial thermal neutron will generate when it is captured (or causes fission)
\item $\epsilon$ is basically 1, but can be slightly larger as it counts the very few fast neutrons that cause fission (off of $^{238}U$)
\item $p$ is the probability that the neutron gets thermalized before being captured 
\item $f$ is the fraction of thermal neutrons that induce fission (some are captured by the moderator)
\eei

In a state where $k$ is close to one, the power generated by the reactor is
\be
P = E_f \frac{dN_f}{dt}
\ee
where $E_f$ is the energy released in one fission cascade ($\sim200$ MeV) and $dN_f/dT$ is the number of neutron-captured fission events per time. 


\chapter{Climate}

\section{Simple Models}

\bei
\item incoming solar radiation: $I_\Sun$=1366 W/m$^2$
\item Earth absorbs $\pi R_\Earth^2$ of this (the cross-sectional area), so total absorbed is $I_\Sun \pi R_\Earth^2$
\item This must equal the radiation emitted from Earth: $4\pi R_\Earth^2 \sigma T_\Earth^4$, so
\item $T_\Earth=278.6K=5.43$C, not bad
\eei 

v2: not all of the radiation from the sun reaches Earth: about 6\% is absorbed by the atmosphere; clouds absorb another 14\% and the surface reflects 10\%. Technically, this is called the {\it albedo}: $a_\Earth=0.3$. Taken at face value, this reduces the total absorbed by $0.7$ and therefore the temperature is smaller by $0.7^{1/4}$, leading to a revised estimate: $T_\Earth=254K=-18$C, much worse.

v3: Account for the fact that some of the radiation emitted by Earth is reflected by the atmosphere back down to Earth, so radiative balance at the surface of the Earth leads to:
\be
4\pi R_\Earth^2 \left[ (1-a_\Earth) I_\Sun/4 + \sigma T_{atm}^4 \right]
= 4\pi R_\Earth^2 \sigma T_\Earth^4
\ee
while radiative balance in the (cartoon-version) atmosphere leads to
\be
T_\Earth^4 = 2T_{atm}^4
\ee
where the factor of 2 accounts for the radiation emitted up and down from the atmosphere. This results in
\be
T_\Earth = \left[ \frac{(1-a_\Earth) I_\Sun}{4\sigma} \big( 1-\frac12\big) \right]^{1/4} = 303K.
\ee
This is the greenhouse effect: radiation from the Earth is trapped and serves to heat it. This assumes that the atmosphere is a single layer and absorbs all of the long wavelength radiation emitted from Earth.


v4: Based on \href{https://brian-rose.github.io/ClimateLaboratoryBook/courseware/elementary-greenhouse.html}{The Climate Laboratory}, section 2, depicted in Fig.~\rf{2layerAtm_sketch}.
\Spng{2layerAtm_sketch}{Simple two-layer model for the temperature.}
Radiative equilibrium must hold at all 3 layers:
\bei
\item Surface: 
\be
\sigma T_s^4 = (1-\alpha) Q + \epsilon\sigma \left( T_0^4 + (1-\epsilon) T_1^4\right)\ee
where $Q$ is the sunlight absorbed on Earth $I_\Sun/4$ and $\epsilon$ is the fraction of short wavelength radiation absorbed by each layer. Note that Kirschoff's Law is used here: the emitted radiation is at the temperature of interest but suppressed by $\epsilon$. If the gas absorbs a lot at the wavelengths of interest, it also emits a lot at those wavelengths.
\item Layer 0:
\be
2\epsilon\sigma T_0^4 = \epsilon\sigma T_s^4 + \epsilon^2\sigma T_1^4\ee
\item Layer 1:
\be
2\epsilon\sigma T_1^4 = (1-\epsilon)\epsilon \sigma T_s^4 + \epsilon^2\sigma T_0^4
\ee
\eei
These are three 3 equations for 4 unknowns (taking $\alpha=0.3$. Fig.~\rf{twolayer} shows the results for the 3 temperatures as a function of the remaining free parameter $\epsilon$.
\Spng{twolayer}{Prediction of the two-layer model for the temperatures of its 3 components as a function of $\epsilon$, the fraction of long wavelength radiation absorbed by the two atmospheric layers.}
The limit $\epsilon=1$ means they absorb all the radiation so the temperatures are identical to the two-layer model in Example 34.2. The limit $\epsilon=0$ corresponds to no greenhouse effect so it recovers the prediction of v2, 254K for the surface.

The horizontal dashed lines are the observed temperatures at the surface, about 3 km up and about 10km up. Clearly, this one-parameter model is not sufficient to fit all 3 data points. Roughly the best fit is at the vertical line, $\epsilon$ a bit lower than 0.6. The temperature there is a tad lower than 300K, getting closer to truth. This model goes beyond v3 in two ways: first, by putting in a second layer, and then by allowing the opacity to drop below 100\%. 

These models are simple because (i) the radiation is not blackbody and the absorbers allow through more radiation at some wavelengths than at others and (ii) radiative transfer is not the only form of heat transfer: convection is also important. 

\section{1D Radiative-Convective Equilibrium}

In equilibrium the mass density $\rho$, pressure $P$, and temperature $T$ can be related to one another. In the 1D case, this become simple:
\bei
\item The upward force on a horizontal slab is $P(z)A-P(z+dz)A=-(dP/dz)A dz$. This is an upward force.This must be balanced by the force of gravity $\rho Adz g$, so
\be
dP = -\rho g dz.
\ee
\item Treating the gas as ideal leads to
\be
P=nk_BT=\frac{\rho}{m} k_BT\ee
\item Therefore,
\be
\frac{dP}{\rho} = \frac{dP k_BT}{mP} = -gdz\ee
or\be
\frac{dP}{P} = -\frac{dz}{H}
\ee
with \be
H\equiv \frac{mg}{k_BT}\ee
\item The solution is therefore
\be
P=P_0e^{-\int_0^z \frac{dz'}{H(z')}}
\ee
\eei

Taking the atmosphere to be composed of some mixture of oxygen (with 32 protons and neutrons) and nitrogen (with 28) for a compromise of 29 times the mass of the nucleon leads to $H=8.4$km at $T=290$K. If the temperature wee constant, the pressure profile would look like Fig.~\rf{pressure}.
\Sfig{pressure}{Pressure profile assuming constant temperature. 290 is the surface temperature and 220 is temperature in the tropopause.}
The actual pressure follows this trend out to about 15km, where the pressure has dropped to about 100-200 mb. But it drops much faster than that at higher altitudes, so more or less moves from the blue to the orange curve.

\subsection{Convective instability}

To determine the pressure and density profile more accurately, we need to determine the temperature profile. For this, we can use the fact that the work done by the pressure of a small volume reduces the internal energy:
\be
dU = -PdV.
\ee
The internal energy drop means a drop in temperature ala $dU=c_VdT$. That is, if the actual drop in temperature is smaller than then the packet will be cooler than the surrounding air and will not rise further. 
Define the actual temperature drop to be {\it lapse rate}
\be
\Gamma \equiv -\frac{dT}{dz}.
\ee
\bee
\item If $|dT|$ from pressure differences is larger than $|\Gamma dz|$, then the packet will be cooler than the surrounding air and it will rise no further. 
\item If $|dT|$ from pressure differences is smaller than $|\Gamma dz|$, then the packet will be hotter than the surrounding air and it will continue to rise, leading to convective instability. 
\eee

The work done can be obtained by taking the differential of the ideal gas law ($PV=Nk_BT)$:
\be
PdV+dPV=Nk_BdT\ee
so
\be
c_VdT= -\left( Nk_BdT-VdP\right).
\ee
From Chapter 5 (5.32), $C_P=C_V+Nk_B$ for a monotomic ideal gas, so
\be
dT = \frac{VdP}{C_P} = \frac{ -\rho Vg dz}{C_P}
\ee
So, there will be convective instability and therefore heat propagation from lower to upper levels if
\be
\frac{ \rho Vg}{C_P} < \Gamma.
\ee
The adiabatic lapse rate \be
\Gamma_d \equiv \frac{ \rho Vg}{C_P}
\ee
is the dividing line in the temperature change: if $\Gamma>\Gamma_d$ then convection occurs until the temperature gradient drops down to $\Gamma_d$. If $\Gamma< \Gamma_d$, then there is no convection.

Near the Earth's surface, we can compute $\Gamma_d$ using the fact that $c_P=C_P/\rho V=1$ kJ/kg-K. We find $\Gamma_d=9.8$ K/km. It might be interesting to compare this to the very simple v3 model above, where the temperature in the ``atmosphere'' was $2^{1/4}=1.19$ smaller than that on Earth. In that model, that means a difference: 
\be
\Delta T = 303\left(1-\frac{1}{1.19}\right)= 48K.\ee
So depending on where you put the top of the atmosphere in that simple model (somewhere between the tropopause at 20km and the stratosphere at 40km), it seems clear that the lapse rate will be roughly of order the adiabatic lapse rate, so convection will play some role.

\end{document}
