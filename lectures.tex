\documentclass[11pt]{book}
\usepackage{graphicx}
\usepackage{amsmath,mathtools}
\begin{document}
\newcommand{\problem}[1]{%\addtocounter{problemc}{1}
\item {#1}
}
\newcommand{\probl}[1]{\label{#1}}
\def\be{\begin{equation}}
\def\ee{\end{equation}}
\def\bea{\begin{eqnarray}}
\def\eea{\end{eqnarray}}
\newcommand{\vs}{\nonumber\\}
\def\across{a^\times}
\def\tcross{T^\times}
\def\ccross{C^\times}
\newcommand{\ec}[1]{Eq.~(\ref{eq:#1})}
\newcommand{\eec}[2]{Eqs.~(\ref{eq:#1}) and (\ref{eq:#2})}
\newcommand{\Ec}[1]{(\ref{eq:#1})}
\newcommand{\eql}[1]{\label{eq:#1}}
\newcommand{\sfig}[2]{
\includegraphics[width=#2]{#1}
        }
\newcommand{\sfigr}[2]{
\includegraphics[angle=270,origin=c,width=#2]{#1}
        }
\newcommand{\sfigra}[2]{
\includegraphics[angle=90,origin=c,width=#2]{#1}
        }
\newcommand{\Sfig}[2]{
   \begin{figure}[thbp]
   \begin{center}
    \sfig{../Figures/#1.pdf}{0.7\columnwidth}
    \caption{{\small #2}}
    \label{fig:#1}
     \end{center}
   \end{figure}
}
\newcommand{\Sfigl}[2]{
   \begin{figure}[thbp]
   \begin{center}
    \sfig{../Figures/#1.pdf}{0.9\columnwidth}
    \caption{{\small #2}}
    \label{fig:#1}
     \end{center}
   \end{figure}
}
\
\newcommand{\Sjpg}[2]{
   \begin{figure}[thbp]
   \begin{center}
    \sfig{../Figures/#1.jpg}{0.8\columnwidth}
    \caption{{\small #2}}
    \label{fig:#1}
     \end{center}
   \end{figure}
}
\newcommand{\Spng}[2]{
   \begin{figure}[thbp]
   \begin{center}
    \sfig{../Figures/#1.png}{0.8\columnwidth}
    \caption{{\small #2}}
    \label{fig:#1}
     \end{center}
   \end{figure}
}

\newcommand{\Sfigr}[2]{
   \begin{figure}[thbp]
   \begin{center}
    \sfigr{../Figures/#1.pdf}{0.5\columnwidth}
    \caption{{\small #2}}
    \label{fig:#1}
     \end{center}
   \end{figure}
}

\newcommand\dirac{\delta_D}
\newcommand{\rf}[1]{\ref{fig:#1}}
\newcommand\example[1]{{\tt EXAMPLE: #1}}
\newcommand\expect[1]{{\tt {\bf Back of the Envelope:} #1}}
\newcommand\theorem[1]{{\tt Theorem: #1}}
\newcommand\bei{\begin{itemize}}
\newcommand\eei{\end{itemize}}
\newcommand\bee{\begin{enumerate}}
\newcommand\eee{\end{enumerate}}
\newcommand\lecture[1]{\newpage
\begin{center}
Lecture #1
\end{center}
}
\newcommand\conversion[1]{\fbox{#1}}

\chapter{Overview}

\bei
\item Annual Per Capita Energy Use is 73 GJ ($10^9$J)
\item
Annual World Energy Use in EJ is 562.1 EJ ($10^{18}$J)
\item
Global Power Use is 17.82 TW ($10^{12}$W)
\item 
Global Energy Consumption is then 156138 TW-hours $=562$ EJ
\item
Daily Energy Use per capita is 200 MJ
\eei
\chapter{Mechanical Energy}

Analyze the energetics of a car trip from Boston to New York. Since the distance is 210 miles and considering a car that gets 30 miles per gallon, the trip requires 7 gallons of gas. 

\conversion{1 Gallon = 3.78 Liters = 3780 cm$^3$}

The density of gas is  0.85 g/cm$^3$. So a gallon of gas has about 3200 gms. The molecule $C_6H_{14}$ has a molar mass of about $6*12+14=86$ gm/mol. So, there is about 37 moles in a gallon of gas. If we estimate the energy per mol as 100 kcal/mol, we get 3700kcal/gallon = 15 MJ/gallon, which is too low by a factor of 8. Perhaps because there are 6 carbon atoms in the molecule, the binding energy is about 800 kcal/mol. The total energy per gallon of gas is 120MJ.

\lecture{1}
\section{Photons}

\end{document}
